% This file is a modified version of the one generated with the
% drexel-thesis LaTeX class created by W. Trevor King. The template
% has been modified to work with Rmarkdown, and specifically,
% bookdown/thesisdown.
%
% This is file `template.tex',
% generated with the docstrip utility.
%
% The original source files were:
%
% drexel-thesis.dtx  (with options: `template')
%
% This is a generated file.
%
% Copyright (C) 2010-2012 W. Trevor King
%
% This file may be distributed and/or modified under the conditions of
% the LaTeX Project Public License, either version 1.3 of this license
% or (at your option) any later version.  The latest version of this
% license is in:
%
%    http://www.latex-project.org/lppl.txt
%
% and version 1.3 or later is part of all distributions of LaTeX version
% 2003/06/01 or later.
%
% This adaptation was done by Tyler Bradley (TB)
%
% Most of the work on the original Reed College document class and template
% was done by Sam Noble (SN). Later comments etc. by Ben Salzberg (BTS).
% Additional restructuring and APA support by Jess Youngberg (JY).
%
% Any line that starts with a percent symbol is a comment.
% They won't show up in the document, and are useful for notes
% to yourself and explaining commands.
% Commenting also removes a line from the document;
% very handy for troubleshooting problems. -BTS

%%
%% Preamble
%%
\documentclass[subfig,blackref,approvalform]{drexel-thesis}
% Packages are extensions to the basic LaTeX functions. Whatever you
% want to typeset, there is probably a package out there for it.
% Chemistry (chemtex), screenplays, you name it.
% Check out CTAN to see: http://www.ctan.org/
%%
\usepackage{graphicx,latexsym}
\usepackage{amsmath}
\usepackage{amssymb,amsthm}
\usepackage{longtable,booktabs,setspace}
\usepackage{chemarr} %% Useful for one reaction arrow, useless if you're not a chem major
%\usepackage[hyphens]{url}
% Added by CII
\usepackage{hyperref}
\usepackage{lmodern}
\usepackage{float}
\floatplacement{figure}{H}
% End of CII addition
\usepackage{rotating}
% Allow pages to be rotated
\usepackage{lscape}

% Next line commented out by CII
%%% \usepackage{natbib}
% Comment out the natbib line above and uncomment the following two lines to use the new
% biblatex-chicago style, for Chicago A. Also make some changes at the end where the
% bibliography is included.
%\usepackage{biblatex-chicago}
%\bibliography{thesis}

% Added by CII (Thanks, Hadley!)
% Use ref for internal links
\renewcommand{\hyperref}[2][???]{\autoref{#1}}
\def\chapterautorefname{Chapter}
\def\sectionautorefname{Section}
\def\subsectionautorefname{Subsection}
% End of CII addition

% Added by CII
\usepackage{caption}
\captionsetup{width=5in}
% End of CII addition

% \usepackage{times} % other fonts are available like times, bookman, charter, palatino

% Syntax highlighting #22
  \usepackage{color}
  \usepackage{fancyvrb}
  \newcommand{\VerbBar}{|}
  \newcommand{\VERB}{\Verb[commandchars=\\\{\}]}
  \DefineVerbatimEnvironment{Highlighting}{Verbatim}{commandchars=\\\{\}}
  % Add ',fontsize=\small' for more characters per line
  \usepackage{framed}
  \definecolor{shadecolor}{RGB}{248,248,248}
  \newenvironment{Shaded}{\begin{snugshade}}{\end{snugshade}}
  \newcommand{\AlertTok}[1]{\textcolor[rgb]{0.94,0.16,0.16}{#1}}
  \newcommand{\AnnotationTok}[1]{\textcolor[rgb]{0.56,0.35,0.01}{\textbf{\textit{#1}}}}
  \newcommand{\AttributeTok}[1]{\textcolor[rgb]{0.77,0.63,0.00}{#1}}
  \newcommand{\BaseNTok}[1]{\textcolor[rgb]{0.00,0.00,0.81}{#1}}
  \newcommand{\BuiltInTok}[1]{#1}
  \newcommand{\CharTok}[1]{\textcolor[rgb]{0.31,0.60,0.02}{#1}}
  \newcommand{\CommentTok}[1]{\textcolor[rgb]{0.56,0.35,0.01}{\textit{#1}}}
  \newcommand{\CommentVarTok}[1]{\textcolor[rgb]{0.56,0.35,0.01}{\textbf{\textit{#1}}}}
  \newcommand{\ConstantTok}[1]{\textcolor[rgb]{0.00,0.00,0.00}{#1}}
  \newcommand{\ControlFlowTok}[1]{\textcolor[rgb]{0.13,0.29,0.53}{\textbf{#1}}}
  \newcommand{\DataTypeTok}[1]{\textcolor[rgb]{0.13,0.29,0.53}{#1}}
  \newcommand{\DecValTok}[1]{\textcolor[rgb]{0.00,0.00,0.81}{#1}}
  \newcommand{\DocumentationTok}[1]{\textcolor[rgb]{0.56,0.35,0.01}{\textbf{\textit{#1}}}}
  \newcommand{\ErrorTok}[1]{\textcolor[rgb]{0.64,0.00,0.00}{\textbf{#1}}}
  \newcommand{\ExtensionTok}[1]{#1}
  \newcommand{\FloatTok}[1]{\textcolor[rgb]{0.00,0.00,0.81}{#1}}
  \newcommand{\FunctionTok}[1]{\textcolor[rgb]{0.00,0.00,0.00}{#1}}
  \newcommand{\ImportTok}[1]{#1}
  \newcommand{\InformationTok}[1]{\textcolor[rgb]{0.56,0.35,0.01}{\textbf{\textit{#1}}}}
  \newcommand{\KeywordTok}[1]{\textcolor[rgb]{0.13,0.29,0.53}{\textbf{#1}}}
  \newcommand{\NormalTok}[1]{#1}
  \newcommand{\OperatorTok}[1]{\textcolor[rgb]{0.81,0.36,0.00}{\textbf{#1}}}
  \newcommand{\OtherTok}[1]{\textcolor[rgb]{0.56,0.35,0.01}{#1}}
  \newcommand{\PreprocessorTok}[1]{\textcolor[rgb]{0.56,0.35,0.01}{\textit{#1}}}
  \newcommand{\RegionMarkerTok}[1]{#1}
  \newcommand{\SpecialCharTok}[1]{\textcolor[rgb]{0.00,0.00,0.00}{#1}}
  \newcommand{\SpecialStringTok}[1]{\textcolor[rgb]{0.31,0.60,0.02}{#1}}
  \newcommand{\StringTok}[1]{\textcolor[rgb]{0.31,0.60,0.02}{#1}}
  \newcommand{\VariableTok}[1]{\textcolor[rgb]{0.00,0.00,0.00}{#1}}
  \newcommand{\VerbatimStringTok}[1]{\textcolor[rgb]{0.31,0.60,0.02}{#1}}
  \newcommand{\WarningTok}[1]{\textcolor[rgb]{0.56,0.35,0.01}{\textbf{\textit{#1}}}}

 \approvalForm{docs/approval-form.pdf}


\author{Your R. Name}    % Fullname
\title{My PhD Dissertation}     % Title Of Thesis
\DUTmonth{May}  % Name of the month of you defense
\DUTyear{2019}   % Year you are defending
\degree{Doctor of Philosophy}    % Your target degree, spelled out

  \advisor{Advisor F. Name}   % Advisor's full name, degree

 \copyrighttext{All Rights Reserved.} % If not "All Rights Reserved."

%% unsrt style give references in order of citation
\bibliographystyle{unsrt}


%If you have two advisors for some reason, you can use the following
% Uncommented out by CII
% TB - commenting this out in hopes for loop will work but this implementation
% is a potential alternative option if it does not
% % End of CII addition



% Added by CII
%%% Copied from knitr
%% maxwidth is the original width if it's less than linewidth
%% otherwise use linewidth (to make sure the graphics do not exceed the margin)
\makeatletter
\def\maxwidth{ %
  \ifdim\Gin@nat@width>\linewidth
    \linewidth
  \else
    \Gin@nat@width
  \fi
}
\makeatother

\renewcommand{\contentsname}{Table of Contents}
% End of CII addition

\setlength{\parskip}{0pt}

% Added by CII

\providecommand{\tightlist}{%
  \setlength{\itemsep}{0pt}\setlength{\parskip}{0pt}}

% Begin document - mostly copied from drexel-thesis template
\begin{document}
\begin{preamble}
\frontmatter
  \begin{dedications} % OPTIONAL
    You can have a dedication here if you wish.
  \end{dedications}
  \begin{acknowledgments} % OPTIONAL
    I want to thank a few people.
  \end{acknowledgments}
  \tableofcontents

  \listoftables

  \listoffigures
  \begin{abstract}
    Here is my abstract
  \end{abstract}
\end{preamble}
\mainmatter
\begin{thesis}
\hypertarget{introduction}{%
\chapter*{Introduction}\label{introduction}}
\addcontentsline{toc}{chapter}{Introduction}

Welcome to the \emph{R Markdown} thesis template. This template is based on (and in many places copied directly from) the Reed College LaTeX template, but hopefully it will provide a nicer interface for those that have never used TeX or LaTeX before. Using \emph{R Markdown} will also allow you to easily keep track of your analyses in \textbf{R} chunks of code, with the resulting plots and output included as well. The hope is this \emph{R Markdown} template gets you in the habit of doing reproducible research, which benefits you long-term as a researcher, but also will greatly help anyone that is trying to reproduce or build onto your results down the road.

Hopefully, you won't have much of a learning period to go through and you will reap the benefits of a nicely formatted thesis. The use of LaTeX in combination with \emph{Markdown} is more consistent than the output of a word processor, much less prone to corruption or crashing, and the resulting file is smaller than a Word file. While you may have never had problems using Word in the past, your thesis is likely going to be about twice as large and complex as anything you've written before, taxing Word's capabilities. After working with \emph{Markdown} and \textbf{R} together for a few weeks, we are confident this will be your reporting style of choice going forward.

\textbf{Why use it?}

\emph{R Markdown} creates a simple and straightforward way to interface with the beauty of LaTeX. Packages have been written in \textbf{R} to work directly with LaTeX to produce nicely formatting tables and paragraphs. In addition to creating a user friendly interface to LaTeX, \emph{R Markdown} also allows you to read in your data, to analyze it and to visualize it using \textbf{R} functions, and also to provide the documentation and commentary on the results of your project. Further, it allows for \textbf{R} results to be passed inline to the commentary of your results. You'll see more on this later.

\textbf{Who should use it?}

Anyone who needs to use data analysis, math, tables, a lot of figures, complex cross-references, or who just cares about the final appearance of their document should use \emph{R Markdown}. Of particular use should be anyone in the sciences, but the user-friendly nature of \emph{Markdown} and its ability to keep track of and easily include figures, automatically generate a table of contents, index, references, table of figures, etc. should make it of great benefit to nearly anyone writing a thesis project.

\textbf{For additional help with bookdown}
Please visit \href{https://bookdown.org/yihui/bookdown/}{the free online bookdown reference guide}.

\hypertarget{first-chapter}{%
\chapter{My First Chapter}\label{first-chapter}}
\begin{Shaded}
\begin{Highlighting}[]
\KeywordTok{library}\NormalTok{(tidyverse)}
\KeywordTok{library}\NormalTok{(readxl)}
\KeywordTok{library}\NormalTok{(knitr)}
\end{Highlighting}
\end{Shaded}
\hypertarget{introduction-1}{%
\section{Introduction}\label{introduction-1}}

Ipsum odio nibh tempus curabitur hendrerit urna dapibus montes magna himenaeos. Quam vivamus odio fermentum quisque imperdiet a vehicula felis dignissim. Etiam montes nulla litora magnis justo himenaeos id diam. Commodo arcu magna ligula varius. Posuere ridiculus nisi vitae fringilla ullamcorper sociosqu dignissim pellentesque cum ridiculus tempus quis eu dictum augue elementum purus mauris.

Sit cum accumsan cras nibh volutpat netus iaculis. Ultricies elementum eget mollis arcu risus habitasse dictumst mi. Potenti ultrices leo sem felis pellentesque conubia ligula orci fames. Netus proin tempor iaculis sollicitudin himenaeos netus etiam nulla varius pharetra. Lacus aptent neque ut congue molestie interdum commodo class placerat molestie cras vitae donec ultricies?
\begin{Shaded}
\begin{Highlighting}[]
\NormalTok{elrod_dat <-}\StringTok{ }\KeywordTok{read_excel}\NormalTok{(}\StringTok{"data/Elrod_2020_2_17_Treated_Virgin.xlsx"}\NormalTok{, }
                        \DataTypeTok{col_names =} \KeywordTok{c}\NormalTok{(}\StringTok{"sample"}\NormalTok{, }\StringTok{"value"}\NormalTok{))}

\KeywordTok{kable}\NormalTok{(}\KeywordTok{head}\NormalTok{(elrod_dat))}
\end{Highlighting}
\end{Shaded}
\begin{tabular}{l|r}
\hline
sample & value\\
\hline
1A & 399.1983\\
\hline
1A & 368.6850\\
\hline
1A & 383.7466\\
\hline
1B & 420.4819\\
\hline
1B & 400.5082\\
\hline
1B & 409.2122\\
\hline
\end{tabular}
\begin{Shaded}
\begin{Highlighting}[]
\NormalTok{elrod_tidy <-}\StringTok{ }\NormalTok{elrod_dat }\OperatorTok\StringTok{ }
\StringTok{  }\KeywordTok{mutate}\NormalTok{(}
    \DataTypeTok{group =} \KeywordTok{str_extract}\NormalTok{(sample, }\StringTok{"[A-C]$"}\NormalTok{),}
    \DataTypeTok{sample =} \KeywordTok{as.numeric}\NormalTok{(}\KeywordTok{str_replace}\NormalTok{(sample, }\StringTok{"[A-C]$"}\NormalTok{, }\StringTok{""}\NormalTok{))}
\NormalTok{  ) }\OperatorTok\StringTok{ }
\StringTok{  }\KeywordTok{group_by}\NormalTok{(group, sample) }\OperatorTok\StringTok{ }
\StringTok{  }\KeywordTok{summarize}\NormalTok{(}
    \DataTypeTok{min =} \KeywordTok{min}\NormalTok{(value),}
    \DataTypeTok{max =} \KeywordTok{max}\NormalTok{(value),}
    \DataTypeTok{median =} \KeywordTok{median}\NormalTok{(value)}
\NormalTok{  ) }

\KeywordTok{kable}\NormalTok{(}\KeywordTok{head}\NormalTok{(elrod_tidy))}
\end{Highlighting}
\end{Shaded}
\begin{tabular}{l|r|r|r|r}
\hline
group & sample & min & max & median\\
\hline
A & 1 & 368.6850 & 399.1983 & 383.7466\\
\hline
A & 2 & 354.3487 & 374.2854 & 368.4673\\
\hline
A & 3 & 385.5298 & 404.0765 & 399.0363\\
\hline
A & 4 & 397.4524 & 417.8496 & 406.6212\\
\hline
A & 5 & 324.7343 & 345.9969 & 332.8127\\
\hline
A & 6 & 268.8548 & 289.2921 & 276.7879\\
\hline
\end{tabular}
\hypertarget{including-plots}{%
\section{Including Plots}\label{including-plots}}

Ipsum fermentum cubilia dictumst pretium sodales pharetra porttitor odio malesuada primis nascetur. Curabitur mus pulvinar nulla egestas ullamcorper class. Mattis rutrum vestibulum orci rutrum litora suspendisse ullamcorper vulputate venenatis suscipit.
\begin{Shaded}
\begin{Highlighting}[]
\NormalTok{elrod_tidy }\OperatorTok\StringTok{ }
\StringTok{  }\KeywordTok{ggplot}\NormalTok{(}\KeywordTok{aes}\NormalTok{(sample, median, }\DataTypeTok{color =}\NormalTok{ group)) }\OperatorTok{+}\StringTok{ }
\StringTok{  }\KeywordTok{geom_point}\NormalTok{() }\OperatorTok{+}
\StringTok{  }\KeywordTok{geom_line}\NormalTok{() }\OperatorTok{+}\StringTok{ }
\StringTok{  }\KeywordTok{geom_errorbar}\NormalTok{(}\KeywordTok{aes}\NormalTok{(}\DataTypeTok{ymin =}\NormalTok{ min, }\DataTypeTok{ymax =}\NormalTok{ max)) }\OperatorTok{+}\StringTok{ }
\StringTok{  }\KeywordTok{theme_bw}\NormalTok{()}
\end{Highlighting}
\end{Shaded}
\includegraphics{thesis_files/figure-latex/elrod-plot-1.pdf}

\hypertarget{using-python}{%
\chapter{Using python}\label{using-python}}

Sit cum accumsan cras nibh volutpat netus iaculis. Ultricies elementum eget mollis arcu risus habitasse dictumst mi. Potenti ultrices leo sem felis pellentesque conubia ligula orci fames. Netus proin tempor iaculis sollicitudin himenaeos netus etiam nulla varius pharetra. Lacus aptent neque ut congue molestie interdum commodo class placerat molestie cras vitae donec ultricies?

\texttt{\{using-python,\ eval\ =\ FALSE\}\ library(reticulate)\ use\_miniconda("/Users/tylerbradley/miniconda2/bin/python")}
\begin{Shaded}
\begin{Highlighting}[]
\ImportTok{import}\NormalTok{ pandas }\ImportTok{as}\NormalTok{ pd}
\end{Highlighting}
\end{Shaded}
\hypertarget{math-sci}{%
\chapter{Mathematics and Science}\label{math-sci}}

\hypertarget{math}{%
\section{Math}\label{math}}

\TeX~is the best way to typeset mathematics. Donald Knuth designed \TeX~when he got frustrated at how long it was taking the typesetters to finish his book, which contained a lot of mathematics. One nice feature of \emph{R Markdown} is its ability to read LaTeX code directly.

If you are doing a thesis that will involve lots of math, you will want to read the following section which has been commented out. If you're not going to use math, skip over or delete this next commented section.

\hypertarget{chemistry-101-symbols}{%
\section{Chemistry 101: Symbols}\label{chemistry-101-symbols}}

Chemical formulas will look best if they are not italicized. Get around math mode's automatic italicizing in LaTeX by using the argument \texttt{\$\textbackslash{}mathrm\{formula\ here\}\$}, with your formula inside the curly brackets. (Notice the use of the backticks here which enclose text that acts as code.)

So, \(\mathrm{Fe_2^{2+}Cr_2O_4}\) is written \texttt{\$\textbackslash{}mathrm\{Fe\_2\^{}\{2+\}Cr\_2O\_4\}\$}.

\noindent Exponent or Superscript: \(\mathrm{O^-}\)

\noindent Subscript: \(\mathrm{CH_4}\)

To stack numbers or letters as in \(\mathrm{Fe_2^{2+}}\), the subscript is defined first, and then the superscript is defined.

\noindent Bullet: CuCl \(\bullet\) \(\mathrm{7H_{2}O}\)

\noindent Delta: \(\Delta\)

\noindent Reaction Arrows: \(\longrightarrow\) or \(\xrightarrow{solution}\)

\noindent Resonance Arrows: \(\leftrightarrow\)

\noindent Reversible Reaction Arrows: \(\rightleftharpoons\)

\hypertarget{typesetting-reactions}{%
\subsection{Typesetting reactions}\label{typesetting-reactions}}

You may wish to put your reaction in an equation environment, which means that LaTeX will place the reaction where it fits and will number the equations for you.
\begin{equation}
  \mathrm{C_6H_{12}O_6  + 6O_2} \longrightarrow \mathrm{6CO_2 + 6H_2O}
  \label{eq:reaction}
\end{equation}
We can reference this combustion of glucose reaction via Equation \eqref{eq:reaction}.

\hypertarget{other-examples-of-reactions}{%
\subsection{Other examples of reactions}\label{other-examples-of-reactions}}

\(\mathrm{NH_4Cl_{(s)}}\) \(\rightleftharpoons\) \(\mathrm{NH_{3(g)}+HCl_{(g)}}\)

\noindent \(\mathrm{MeCH_2Br + Mg}\) \(\xrightarrow[below]{above}\) \(\mathrm{MeCH_2\bullet Mg \bullet Br}\)

\hypertarget{physics}{%
\section{Physics}\label{physics}}

Many of the symbols you will need can be found on the math page \url{http://web.reed.edu/cis/help/latex/math.html} and the Comprehensive LaTeX Symbol Guide (\url{http://mirror.utexas.edu/ctan/info/symbols/comprehensive/symbols-letter.pdf}).

\hypertarget{biology}{%
\section{Biology}\label{biology}}

You will probably find the resources at \url{http://www.lecb.ncifcrf.gov/~toms/latex.html} helpful, particularly the links to bsts for various journals. You may also be interested in TeXShade for nucleotide typesetting (\url{http://homepages.uni-tuebingen.de/beitz/txe.html}). Be sure to read the proceeding chapter on graphics and tables.

\hypertarget{ref-labels}{%
\chapter{Tables, Graphics, References, and Labels}\label{ref-labels}}

\hypertarget{tables}{%
\section{Tables}\label{tables}}

In addition to the tables that can be automatically generated from a data frame in \textbf{R} that you saw in {[}R Markdown Basics{]} using the \texttt{kable} function, you can also create tables using \emph{pandoc}. (More information is available at \url{http://pandoc.org/README.html\#tables}.) This might be useful if you don't have values specifically stored in \textbf{R}, but you'd like to display them in table form. Below is an example. Pay careful attention to the alignment in the table and hyphens to create the rows and columns.
\begin{longtable}[]{@{}ccc@{}}
\caption{\label{tab:inher} Correlation of Inheritance Factors for Parents and Child}\tabularnewline
\toprule
\begin{minipage}[b]{0.29\columnwidth}\centering
Factors\strut
\end{minipage} & \begin{minipage}[b]{0.46\columnwidth}\centering
Correlation between Parents \& Child\strut
\end{minipage} & \begin{minipage}[b]{0.16\columnwidth}\centering
Inherited\strut
\end{minipage}\tabularnewline
\midrule
\endfirsthead
\toprule
\begin{minipage}[b]{0.29\columnwidth}\centering
Factors\strut
\end{minipage} & \begin{minipage}[b]{0.46\columnwidth}\centering
Correlation between Parents \& Child\strut
\end{minipage} & \begin{minipage}[b]{0.16\columnwidth}\centering
Inherited\strut
\end{minipage}\tabularnewline
\midrule
\endhead
\begin{minipage}[t]{0.29\columnwidth}\centering
Education\strut
\end{minipage} & \begin{minipage}[t]{0.46\columnwidth}\centering
-0.49\strut
\end{minipage} & \begin{minipage}[t]{0.16\columnwidth}\centering
Yes\strut
\end{minipage}\tabularnewline
\begin{minipage}[t]{0.29\columnwidth}\centering
Socio-Economic Status\strut
\end{minipage} & \begin{minipage}[t]{0.46\columnwidth}\centering
0.28\strut
\end{minipage} & \begin{minipage}[t]{0.16\columnwidth}\centering
Slight\strut
\end{minipage}\tabularnewline
\begin{minipage}[t]{0.29\columnwidth}\centering
Income\strut
\end{minipage} & \begin{minipage}[t]{0.46\columnwidth}\centering
0.08\strut
\end{minipage} & \begin{minipage}[t]{0.16\columnwidth}\centering
No\strut
\end{minipage}\tabularnewline
\begin{minipage}[t]{0.29\columnwidth}\centering
Family Size\strut
\end{minipage} & \begin{minipage}[t]{0.46\columnwidth}\centering
0.18\strut
\end{minipage} & \begin{minipage}[t]{0.16\columnwidth}\centering
Slight\strut
\end{minipage}\tabularnewline
\begin{minipage}[t]{0.29\columnwidth}\centering
Occupational Prestige\strut
\end{minipage} & \begin{minipage}[t]{0.46\columnwidth}\centering
0.21\strut
\end{minipage} & \begin{minipage}[t]{0.16\columnwidth}\centering
Slight\strut
\end{minipage}\tabularnewline
\bottomrule
\end{longtable}
We can also create a link to the table by doing the following: Table \ref{tab:inher}. If you go back to {[}Loading and exploring data{]} and look at the \texttt{kable} table, we can create a reference to this max delays table too: Table \ref{tab:maxdelays}. The addition of the \texttt{(\textbackslash{}\#tab:inher)} option to the end of the table caption allows us to then make a reference to Table \texttt{\textbackslash{}@ref(tab:label)}. Note that this reference could appear anywhere throughout the document after the table has appeared.

\clearpage

\hypertarget{figures}{%
\section{Figures}\label{figures}}

If your thesis has a lot of figures, \emph{R Markdown} might behave better for you than that other word processor. One perk is that it will automatically number the figures accordingly in each chapter. You'll also be able to create a label for each figure, add a caption, and then reference the figure in a way similar to what we saw with tables earlier. If you label your figures, you can move the figures around and \emph{R Markdown} will automatically adjust the numbering for you. No need for you to remember! So that you don't have to get too far into LaTeX to do this, a couple \textbf{R} functions have been created for you to assist. You'll see their use below.

In the \textbf{R} chunk below, we will load in a picture stored as \texttt{reed.jpg} in our main directory. We then give it the caption of ``Reed logo'', the label of ``reedlogo'', and specify that this is a figure. Make note of the different \textbf{R} chunk options that are given in the R Markdown file (not shown in the knitted document).
\begin{Shaded}
\begin{Highlighting}[]
\KeywordTok{include_graphics}\NormalTok{(}\DataTypeTok{path =} \StringTok{"figure/drexel-logo.pdf"}\NormalTok{)}
\end{Highlighting}
\end{Shaded}
\begin{figure}
\centering
\includegraphics{figure/drexel-logo.pdf}
\caption{\label{fig:drexellogo}Drexel logo}
\end{figure}
Here is a reference to the Drexel logo: Figure \ref{fig:drexellogo}. Note the use of the \texttt{fig:} code here. By naming the \textbf{R} chunk that contains the figure, we can then reference that figure later as done in the first sentence here. We can also specify the caption for the figure via the R chunk option \texttt{fig.cap}.

\clearpage

Below we will investigate how to save the output of an \textbf{R} plot and label it in a way similar to that done above. Recall the \texttt{flights} dataset from Chapter \ref{rmd-basics}. (Note that we've shown a different way to reference a section or chapter here.) We will next explore a bar graph with the mean flight departure delays by airline from Portland for 2014. Note also the use of the \texttt{scale} parameter which is discussed on the next page.
\begin{Shaded}
\begin{Highlighting}[]
\NormalTok{flights }\OperatorTok\StringTok{ }\KeywordTok{group_by}\NormalTok{(carrier) }\OperatorTok
\StringTok{  }\KeywordTok{summarize}\NormalTok{(}\DataTypeTok{mean_dep_delay =} \KeywordTok{mean}\NormalTok{(dep_delay)) }\OperatorTok
\StringTok{  }\KeywordTok{ggplot}\NormalTok{(}\KeywordTok{aes}\NormalTok{(}\DataTypeTok{x =}\NormalTok{ carrier, }\DataTypeTok{y =}\NormalTok{ mean_dep_delay)) }\OperatorTok{+}
\StringTok{  }\KeywordTok{geom_bar}\NormalTok{(}\DataTypeTok{position =} \StringTok{"identity"}\NormalTok{, }\DataTypeTok{stat =} \StringTok{"identity"}\NormalTok{, }\DataTypeTok{fill =} \StringTok{"red"}\NormalTok{)}
\end{Highlighting}
\end{Shaded}
\begin{figure}
\centering
\includegraphics{thesis_files/figure-latex/delaysboxplot-1.pdf}
\caption{\label{fig:delaysboxplot}Mean Delays by Airline}
\end{figure}
Here is a reference to this image: Figure \ref{fig:delaysboxplot}.

A table linking these carrier codes to airline names is available at \url{https://github.com/ismayc/pnwflights14/blob/master/data/airlines.csv}.

\clearpage

Next, we will explore the use of the \texttt{out.extra} chunk option, which can be used to shrink or expand an image loaded from a file by specifying \texttt{"scale=\ "}. Here we use the mathematical graph stored in the ``subdivision.pdf'' file.
\begin{figure}
\includegraphics[scale=0.75]{figure/subdivision} \caption{Subdiv. graph}\label{fig:subd}
\end{figure}
Here is a reference to this image: Figure \ref{fig:subd}. Note that \texttt{echo=FALSE} is specified so that the \textbf{R} code is hidden in the document.

\textbf{More Figure Stuff}

Lastly, we will explore how to rotate and enlarge figures using the \texttt{out.extra} chunk option. (Currently this only works in the PDF version of the book.)
\begin{figure}
\includegraphics[angle=180, scale=1.1]{figure/subdivision} \caption{A Larger Figure, Flipped Upside Down}\label{fig:subd2}
\end{figure}
As another example, here is a reference: Figure \ref{fig:subd2}.

\hypertarget{footnotes-and-endnotes}{%
\section{Footnotes and Endnotes}\label{footnotes-and-endnotes}}

You might want to footnote something.\footnote{footnote text} The footnote will be in a smaller font and placed appropriately. Endnotes work in much the same way.

\hypertarget{bibliographies}{%
\section{Bibliographies}\label{bibliographies}}

Of course you will need to cite things, and you will probably accumulate an armful of sources. There are a variety of tools available for creating a bibliography database (stored with the .bib extension). In addition to BibTeX suggested below, you may want to consider using the free and easy-to-use tool called Zotero. In addition, a tutorial is available from Middlebury College at \url{http://sites.middlebury.edu/zoteromiddlebury/}.

\emph{R Markdown} uses \emph{pandoc} (\url{http://pandoc.org/}) to build its bibliographies. One nice caveat of this is that you won't have to do a second compile to load in references as standard LaTeX requires. To cite references in your thesis (after creating your bibliography database), place the reference name inside square brackets and precede it by the ``at'' symbol. For example, here's a reference to a book about worrying: (Molina \& Borkovec, 1994). This \texttt{Molina1994} entry appears in a file called \texttt{thesis.bib} in the \texttt{bib} folder. This bibliography database file was created by a program called BibTeX. You can call this file something else if you like (look at the YAML header in the main .Rmd file) and, by default, is to placed in the \texttt{bib} folder.

For more information about BibTeX and bibliographies, see our CUS site (\url{http://web.reed.edu/cis/help/latex/index.html})\footnote{Reed~College (2007)}. There are three pages on this topic: \emph{bibtex} (which talks about using BibTeX, at \url{http://web.reed.edu/cis/help/latex/bibtex.html}), \emph{bibtexstyles} (about how to find and use the bibliography style that best suits your needs, at \url{http://web.reed.edu/cis/help/latex/bibtexstyles.html}) and \emph{bibman} (which covers how to make and maintain a bibliography by hand, without BibTeX, at \url{http://web.reed.edu/cis/help/latex/bibman.html}). The last page will not be useful unless you have only a few sources.

If you look at the YAML header at the top of the main .Rmd file you can see that we can specify the style of the bibliography by referencing the appropriate csl file. You can download a variety of different style files at \url{https://www.zotero.org/styles}. Make sure to download the file into the csl folder.

\textbf{Tips for Bibliographies}
\begin{itemize}
\tightlist
\item
  Like with thesis formatting, the sooner you start compiling your bibliography for something as large as thesis, the better. Typing in source after source is mind-numbing enough; do you really want to do it for hours on end in late April? Think of it as procrastination.
\item
  The cite key (a citation's label) needs to be unique from the other entries.
\item
  When you have more than one author or editor, you need to separate each author's name by the word ``and'' e.g. \texttt{Author\ =\ \{Noble,\ Sam\ and\ Youngberg,\ Jessica\},}.
\item
  Bibliographies made using BibTeX (whether manually or using a manager) accept LaTeX markup, so you can italicize and add symbols as necessary.
\item
  To force capitalization in an article title or where all lowercase is generally used, bracket the capital letter in curly braces.
\end{itemize}
\hypertarget{anything-else}{%
\section{Anything else?}\label{anything-else}}

If you'd like to see examples of other things in this template, please file an issue at the \href{https://github.com/tbradley1013/dragondown}{\texttt{dragondown} github repo}

\hypertarget{conclusion}{%
\chapter*{Conclusion}\label{conclusion}}
\addcontentsline{toc}{chapter}{Conclusion}

If we don't want Conclusion to have a chapter number next to it, we can add the \texttt{\{-\}} attribute.

\textbf{More info}

And here's some other random info: the first paragraph after a chapter title or section head \emph{shouldn't be} indented, because indents are to tell the reader that you're starting a new paragraph. Since that's obvious after a chapter or section title, proper typesetting doesn't add an indent there.

\backmatter

\hypertarget{list-of-references}{%
\chapter*{List of References}\label{list-of-references}}
\addcontentsline{toc}{chapter}{List of References}

\markboth{References}{References}

\noindent

\setlength{\parindent}{-0.20in}
\setlength{\leftskip}{0.20in}
\setlength{\parskip}{8pt}

\hypertarget{refs}{}
\leavevmode\hypertarget{ref-angel2000}{}%
Angel, E. (2000). \emph{Interactive computer graphics : A top-down approach with opengl}. Boston, MA: Addison Wesley Longman.

\leavevmode\hypertarget{ref-angel2001}{}%
Angel, E. (2001a). \emph{Batch-file computer graphics : A bottom-up approach with quicktime}. Boston, MA: Wesley Addison Longman.

\leavevmode\hypertarget{ref-angel2002a}{}%
Angel, E. (2001b). \emph{Test second book by angel}. Boston, MA: Wesley Addison Longman.

\leavevmode\hypertarget{ref-Molina1994}{}%
Molina, S. T., \& Borkovec, T. D. (1994). The Penn State worry questionnaire: Psychometric properties and associated characteristics. In G. C. L. Davey \& F. Tallis (Eds.), \emph{Worrying: Perspectives on theory, assessment and treatment} (pp. 265--283). New York: Wiley.

\leavevmode\hypertarget{ref-reedweb2007}{}%
Reed~College. (2007). LaTeX your document. Retrieved from \url{http://web.reed.edu/cis/help/LaTeX/index.html}

\appendix

\hypertarget{appendix-a-the-first-appendix}{%
\chapter{Appendix A: The First Appendix}\label{appendix-a-the-first-appendix}}

This first appendix includes all of the R chunks of code that were hidden throughout the document (using the \texttt{include\ =\ FALSE} chunk tag) to help with readibility and/or setup.

\textbf{In the main Rmd file}
\begin{Shaded}
\begin{Highlighting}[]
\CommentTok{# This chunk ensures that the dragondown package is}
\CommentTok{# installed and loaded. This dragondown package includes}
\CommentTok{# the template files for the thesis.}
\ControlFlowTok{if}\NormalTok{(}\OperatorTok{!}\KeywordTok{require}\NormalTok{(devtools))}
  \KeywordTok{install.packages}\NormalTok{(}\StringTok{"devtools"}\NormalTok{, }\DataTypeTok{repos =} \StringTok{"http://cran.rstudio.com"}\NormalTok{)}
\ControlFlowTok{if}\NormalTok{(}\OperatorTok{!}\KeywordTok{require}\NormalTok{(dragondown))}
\NormalTok{  devtools}\OperatorTok{::}\KeywordTok{install_github}\NormalTok{(}\StringTok{"tbradley1013/dragondown"}\NormalTok{)}
\KeywordTok{library}\NormalTok{(dragondown)}
\end{Highlighting}
\end{Shaded}
\textbf{In Chapter \ref{ref-labels}:}
\begin{Shaded}
\begin{Highlighting}[]
\CommentTok{# This chunk ensures that the dragondown package is}
\CommentTok{# installed and loaded. This dragondown package includes}
\CommentTok{# the template files for the thesis and also two functions}
\CommentTok{# used for labeling and referencing}
\ControlFlowTok{if}\NormalTok{(}\OperatorTok{!}\KeywordTok{require}\NormalTok{(devtools))}
  \KeywordTok{install.packages}\NormalTok{(}\StringTok{"devtools"}\NormalTok{, }\DataTypeTok{repos =} \StringTok{"http://cran.rstudio.com"}\NormalTok{)}
\ControlFlowTok{if}\NormalTok{(}\OperatorTok{!}\KeywordTok{require}\NormalTok{(dplyr))}
    \KeywordTok{install.packages}\NormalTok{(}\StringTok{"dplyr"}\NormalTok{, }\DataTypeTok{repos =} \StringTok{"http://cran.rstudio.com"}\NormalTok{)}
\ControlFlowTok{if}\NormalTok{(}\OperatorTok{!}\KeywordTok{require}\NormalTok{(ggplot2))}
    \KeywordTok{install.packages}\NormalTok{(}\StringTok{"ggplot2"}\NormalTok{, }\DataTypeTok{repos =} \StringTok{"http://cran.rstudio.com"}\NormalTok{)}
\ControlFlowTok{if}\NormalTok{(}\OperatorTok{!}\KeywordTok{require}\NormalTok{(ggplot2))}
    \KeywordTok{install.packages}\NormalTok{(}\StringTok{"bookdown"}\NormalTok{, }\DataTypeTok{repos =} \StringTok{"http://cran.rstudio.com"}\NormalTok{)}
\ControlFlowTok{if}\NormalTok{(}\OperatorTok{!}\KeywordTok{require}\NormalTok{(dragondown))\{}
  \KeywordTok{library}\NormalTok{(devtools)}
\NormalTok{  devtools}\OperatorTok{::}\KeywordTok{install_github}\NormalTok{(}\StringTok{"tbradley1013/dragondown"}\NormalTok{)}
\NormalTok{  \}}
\KeywordTok{library}\NormalTok{(dragondown)}
\NormalTok{flights <-}\StringTok{ }\KeywordTok{read.csv}\NormalTok{(}\StringTok{"data/flights.csv"}\NormalTok{)}
\end{Highlighting}
\end{Shaded}
\hypertarget{appendix-b-the-second-appendix-for-fun}{%
\chapter{Appendix B: The Second Appendix, for Fun}\label{appendix-b-the-second-appendix-for-fun}}
\end{thesis}
%\bibliography{references} % Include references.bib BibTeX

%%% \appendix
  \begin{vita} % Ph.D. only.
    The Vita is a biography of the student written in the third person in either essay or outline form. It should include full name, place and date of birth (optional), country of citizenship (optional), institutions attended, degrees and honors received, titles of publications, teaching and professional experience, and other pertinent data. The vita should be one page or less
  \end{vita}
\end{document}
\endinput



%%%%%%%%%%%%%%%%%%%%%%%%%%%%%%%%%%%%%%%%%%



% \frontmatter % this stuff will be roman-numbered
% \pagestyle{empty} % this removes page numbers from the frontmatter
%
% \mainmatter % here the regular arabic numbering starts
% \pagestyle{fancyplain} % turns page numbering back on
